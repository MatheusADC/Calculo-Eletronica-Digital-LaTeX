\documentclass[12pt,openany,oneside,a4paper]{abntex2}
\usepackage[portuguese]{babel}
\usepackage{amsmath}
\usepackage{graphicx}
\usepackage{lipsum}
\usepackage{circuitikz}
\usepackage{icomma}
\usepackage{tikz}
\usepackage{float}
\usepackage{circuitikz}

\titulo{Análise de um Circuito de Pequenos Sinais}
\autor{Matheus Amaral da Costa}
\data{Março de 2025}
\instituicao{
  Centro Universitário FAESA
  \par
  Engenharia da Computação}
\local{Vitória, ES}

\begin{document}

\imprimircapa
\imprimirfolhaderosto*

\tableofcontents

\chapter{Análise DC}
\section{Considerações Iniciais}
Considerar que os capacitores estão "abertos".

\section{Representação do Circuito}
\begin{center}
    \begin{circuitikz}[american]
        \draw (3,2) node[npn, anchor=B, scale=1.5](Q1){};
        \draw (0,0) to node[ground]{} (0,0) to[sV, a=$v_{in}$,v=$1mV$] (0,2)
          to[C] (Q1.B);
        \draw (Q1.B) to[R, , a=$R_1$,l=$10k\Omega$]
              ++(0,3) -- ++(0.5,0) -- ++(0,0.5)
              node[ocirc](Vcc){$+10_{Vcc}$} 
              (Vcc) ++(0,-0.5) -- ++(0.75,0) to[R, a=$R_C$, l=$3{,}6k\Omega$] (Q1.C);
        \draw (Q1.C) to[C] (8,3.2) to[R, a=$R_L$,l=$10k\Omega$] (8,0)
              node[ground]{};
        \draw (Q1.B) to[R, l=$R_2$, a=$2{,}2k\Omega$] (3,0)
              node[ground]{};
        \draw (Q1.E) -- (4.3, 0) to[R, a=$R_E$, l=$1k\Omega$] (4.3,-2) node[ground]{};
        \draw (Q1.E) to[C] (6.3,0.88);
        \draw (6.3,0.88) -- (6.3,-1) node[ground]{};
        \node at (4,-3) {\large \textbf{Figura 1 - Circuito Elétrico}};
    \end{circuitikz}
\end{center}

\section{Dados do Circuito}
\begin{itemize}
    \item $V_{CC} = 10V$
    \item $R_1 = 10\,\text{k}\Omega$
    \item $R_2 = 2{,}2\,\text{k}\Omega$
    \item $R_C = 3{,}6 \, \text{k}\Omega$
    \item $R_E = 1 \, \text{k}\Omega$
    \item $R_L = 10 \, \text{k}\Omega$
\end{itemize}

\section{Determinação de $V_B$}
\[
V_B = V_{CC} \times \left(\frac{R_2}{R_1 + R_2}\right)
\]
\[
V_B = 10 \times (\frac{2{,}2 \times 10^3}{12{,}2 \times 10^3})
\]
\[
\tikz[baseline]{\node[draw=red, thick, rounded corners] {$V_B = 1{,}8\,V$};}
\]

\section{Cálculo de $I_E$}
Aplicando LKT na malha $\alpha$, tem-se:
\[
V_B - V_{BE} - R_E \cdot I_E = 0
\]
Se o diodo é de Silício, tem-se que $V_{BE} = 0{,}7V$, logo:
\[
1{,}8 - 0{,}7 - 1000 \cdot I = 0 \implies I_E = \frac{1{,}1}{1000} \implies \tikz[baseline]{\node[draw=red, thick, rounded corners] {$I_E = 1{,}1 \, \text{mA}$};}
\]

\section{Cálculo de $I_C$}
Para $\beta >= 100$, temos que $I_C \approx I_E \implies \tikz[baseline]{\node[draw=red, thick, rounded corners] {$I_C = 1{,}1 \, \text{mA}$};}$

\section{Cálculo de $I_B$}
Adotando $\beta$ = 100, temos: 
\[
I_B = \frac{I_C}{\beta} = \frac{1{,}1 \times 10^3}{100} \implies \tikz[baseline]{\node[draw=red, thick, rounded corners] {$I_B = 11 \, \text{µA}$};}
\]

\section{Cálculo do $I_C$ de Saturação}
\[
I_{Cmáx} = \frac{V_{CC}}{R_C + R_E} = \frac{10}{4{,}6 \times 10^3} \implies \tikz[baseline]{\node[draw=red, thick, rounded corners] {$I_{Cmáx} = 2{,}17 \, \text{mA}$};}
\]

\section{Cálculo do $V_{CE}$}
\[
V_{CE} = V_{CC} - R_C I_C - R_E I_E
\]
\[
V_{CE} = 10 - (R_C + R_E) \cdot 1{,}1 \times 10^{-3} \implies V_{CE} = 10 - 5{,}06 \implies \tikz[baseline]{\node[draw=red, thick, rounded corners] {$V_{CE} = 4{,}94 \, \text{V}$};}
\]

\section{Gráfico $I_{C}$ x $V_{CE}$}
\begin{center}
    \begin{tikzpicture}
        \draw[->] (0,0) -- (5,0) node[right] {$V_{CE}$};
        \draw[->] (0,0) -- (0,5) node[above] {$I_C$};
        \draw[thick] (4,0) -- (0,4) node[left] {Saturação};
        \node at (4,-0.3) {Corte};
        \node at (3,2.3) {Operação};
        \node at (2.5,-1) {\large \textbf{Figura 2 - Gráfico $I_C$ x $V_{CE}$}};
    \end{tikzpicture}
\end{center}

\section{Conclusão}
Devido ao fato de $I_{C}$ ser menor que 2{,}17 mA ($I_{C\text{máx}}$) e ser diferente de zero, a configuração está funcionando como um amplificador.

\chapter{Análise AC}
\section{Considerações Iniciais}
Considerar os capacitores em curto-circuito; \\
Aterrar as fontes contínuas.

\section{Simplificação do Circuito (Modelo Pi)}
\begin{center}
    \begin{circuitikz}[american]
        \draw (-1,0) node[ground]{} to[sV, a=$v_{in}$, l=$1mV$] (-1,3);
        \draw (-1,3) -- (1.5,3) 
        to[R, a=$2{,}2k\Omega$,l=$R_{2}$] (1.5,0) node[ground]{};
        \draw (1.5,3) -- (4,3) 
        to[R, a=$10k\Omega$,l=$R_{1}$] (4,0) node[ground]{};
        \draw (4,3) -- (6,3) 
        to[R, l=$\beta re'$] (6,0);
        \draw (6,0) -- (7,0) node[ground]{};
        \draw (7,0) -- (8,0)
         to[isource, a=$i_c$, invert] (8,3)
         -- (10.5,3);
        \draw (10.5,3) -- (10.5,3) 
        to[R, a=$3{,}6k\Omega$, l=$R_C$] (10.5,0) node[ground]{};
        \draw (10.5,3) -- (13,3) 
        to[R, a=$10k\Omega$,l=$R_L$] (13,0) node[ground]{};
        \draw[dashed] (9,-1) rectangle (14,3.5);
        \node at (11,3.8) {$r_c$};
    \end{circuitikz}
    \vspace{0.5cm}
    {\large \textbf{Figura 3 - Circuito Simplificado (Modelo Pi)}}
\end{center}

\section{Cálculo de $r_c$}
\[
r_c = {R_C \parallel R_L} \implies r_c = \frac{R_C \cdot R_L}{R_C + R_L}
\]
\[
r_c = \frac{10 \, \text{k}\Omega \times 3{,}6 \, \text{k}\Omega}{10 \, \text{k}\Omega + 3{,}6 \, \text{k}\Omega}
\]
\[
\tikz[baseline]{\node[draw=red, thick, rounded corners] {$r_c = 2{,}65 \, \text{k}\Omega$};}
\]

\section{Cálculo de $r_e'$}
\[
r_e^{\prime} = \frac{25 \, \text{mV}}{I_E} = \frac{25 \times 10^{-3}}{1{,}1 \times 10^{-3}} \implies \tikz[baseline]{\node[draw=red, thick, rounded corners] {$r_e^{\prime} = 22{,}72 \, \Omega$};}
\]

\section{Cálculo do Amplificador de Tensão}
\[
A_v = \frac{r_c}{r_e'} = \frac{2{,}65 \times 10^3}{22{,}72} \implies \tikz[baseline]{\node[draw=red, thick, rounded corners] {$A_v = 116{,}64$};}
\]

\section{Cálculo do ganho (em decibéis) da configuração}
\[
A_{v\text{dB}} = 20 \log_{10} A_v \implies A_{v\text{dB}} = 20 \log_{10} 116{,}64 \implies \tikz[baseline]{\node[draw=red, thick, rounded corners] {$A_{v\text{dB}} = 41{,}3 \, \text{dB}$};}
\]

\end{document}
